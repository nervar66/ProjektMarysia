\documentclass[a4paper,twoside,12pt]{mgr}
\makeatletter
\def\@cite#1#2{{#1\if@tempswa , #2\fi}}
\makeatother
\usepackage{fancyhdr}
\pagestyle{fancy}
\usepackage{cite}
\usepackage{polski}
\usepackage[utf8]{inputenc}
\usepackage{float}
\restylefloat{figure}
\usepackage{listingsutf8}
\usepackage{color}
\usepackage{hyperref}
\usepackage{textcomp}
\definecolor{listinggray}{gray}{0.9}
\definecolor{lbcolor}{rgb}{0.9,0.9,0.9}

\lstset{
    inputencoding=utf8,
	backgroundcolor=\color{lbcolor},
	tabsize=4,
	rulecolor=,
	language=java,
        basicstyle=\scriptsize,
        upquote=true,
        aboveskip={1.5\baselineskip},
        columns=fixed,
        showstringspaces=false,
        extendedchars=true,
        breaklines=true,
        prebreak = \raisebox{0ex}[0ex][0ex]{\ensuremath{\hookleftarrow}},
        frame=single,
        showtabs=false,
        showspaces=false,
        showstringspaces=false,
        identifierstyle=\ttfamily,
        keywordstyle=\color[rgb]{0,0,1},
        commentstyle=\color[rgb]{0.133,0.545,0.133},
        stringstyle=\color[rgb]{0.627,0.126,0.941},
}
\usepackage{graphicx}
\fancyhf{}
\fancyhead[LE,LO]{\leftmark}
\fancyfoot[CE,CO]{- \thepage\ -}
%\linespread{1.3}
\fancypagestyle{plain}{
\fancyhead[LE,LO]{\leftmark}
\fancyfoot[CE,CO]{- \thepage\ -} 
}
\raggedbottom


%**************************************************************************
% Dane do strony tytułowej
%

% Autor
\autor{Piotr Zyszczak i Damian Łukasik}

% Rodzaj pracy - wpisać LICENCJACKA, INŻYNIERSKA lub MAGISTERSKA
\rodzajPracy{}

% Tytuł pracy magisterskiej/inżynierskiej
\tytul{Protokół Kerberos}

% Rok
\rok{2015}

% Kierunek
\kierunek{Informatyka}

% Studia stacjonarne lub niestacjonarne (wpisać jakie)
\studia{stacjonarne}

% Poziom studiów wpisać I lub II
\poziomStudiow{II}


% Numer albumu
\numerAlbumu{113066/112993}

%
%**************************************************************************


% Styl dla wtrąceń anglojęzycznych
\newcommand{\eng}[1]{(\emph{#1})}

\begin{document}

\stronaTytulowa

\tableofcontents
\chapter{Cel i zakres zajęć}
Zademonstrowanie praktyczne systemu Kerberos w działaniu.

\chapter{Wstęp teoretyczny}
Ze strony \url{http://www-01.ibm.com/support/knowledgecenter/ssw_aix_61/com.ibm.aix.security/kerberos_intro.htm?lang=pl}

Protokół Kerberos jest usługą uwierzytelniania sieciowego umożliwiającą weryfikację tożsamości nazw użytkowników w sieciach fizycznych, które nie są zabezpieczone. Kerberos zapewnia uwierzytelnianie wzajemne, integrację danych oraz prywatność przy założeniu, że ruch w sieci jest wrażliwy na przechwytywanie, kontrolowanie i zastępowanie.

Nazwa użytkownika Kerberos jest unikalnym identyfikatorem używającym usług uwierzytelniania Kerberos. Protokół Kerberos sprawdza tożsamości bez polegania na uwierzytelnianiu przez system operacyjny hosta, za to bazuje na adresach hostów lub wymaga fizycznego bezpieczeństwa wszystkich hostów w sieci.

Bilety Kerberos uwiarygadniają tożsamość. Są dwa typy biletów: bilet przydzielania biletu oraz bilet usługi. Bilet przydzielania biletu jest używany w początkowym żądaniu weryfikacji tożsamości. Podczas logowania do hosta potrzebna jest weryfikacja tożsamości, taka jak hasło lub token. Po uzyskaniu biletu przydzielania biletu można go użyć w celu zażądania biletów dla konkretnych usług. Metoda dwóch biletów jest znana jako zaufana osoba trzecia Kerberos. Bilet przydzielania biletu uwierzytelnia użytkownika w serwerze Kerberos, a bilet usługi stanowi bezpieczne wprowadzenie do usługi.

Zaufana osoba trzecia lub pośrednik w uwierzytelnianiu Kerberos nosi nazwę Centrum dystrybucji kluczy (Key Distribution Center - KDC). KDC wystawia dla klientów wszystkie bilety Kerberos.

\chapter{Przebieg}
Krok pierwszy to utworzenie przystawki Użytkownicy i komputery usługi Active Directory, znajdujemy węzeł Computers oraz odpowiedni komputer. Otwieramy Właściwości oraz przechodzimy do zakładki

Dalej są dwie opcje:
\begin{itemize}
\item Ufaj temu komputerowi w delegowaniu do dowolnej usługi (tylko Kerberos)
\item Ufaj temu komputerowi w delegowaniu tylko do określonych usług
\end{itemize}
Przy wyborze pierwszej możliwe jest delegowanie do wszystkich usług używając protokołu Kerberos,
natomiast w drugiej opcji wybieramy usługi oraz protokół jakiego chcemy używać.

Dla punktu drugiego, zaznaczamy Ufaj temu komputerowi w delegowaniu tylko do określonych usług,
następnie użyto usługi Kerberos i naciśnięto Dodaj i w oknie Dodawanie usług naciśnięto Użytkownicy i
komputery. Wyszukujemy odpowiedniego użytkownika lub komputer w polu Wprowadź nazwy obiektów do
wybrania.

Pozostaje tylko wybrać usługi które obdarzymy zaufaniem.
\chapter{Wnioski}
Kerberos jest wygodnym i łatwym do zainstalowania protokołem.

\end{document}
