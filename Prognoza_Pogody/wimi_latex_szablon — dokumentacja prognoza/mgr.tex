\documentclass[a4paper,twoside,12pt]{mgr}
\makeatletter
\def\@cite#1#2{{#1\if@tempswa , #2\fi}}
\makeatother
\usepackage{fancyhdr}
\pagestyle{fancy}
\usepackage{cite}
\usepackage{polski}
\usepackage[utf8]{inputenc}
\usepackage{float}
\restylefloat{figure}
\usepackage{listingsutf8}
\usepackage{color}
\usepackage{hyperref}
\usepackage{textcomp}
\definecolor{listinggray}{gray}{0.9}
\definecolor{lbcolor}{rgb}{0.9,0.9,0.9}

\lstset{
    inputencoding=utf8,
	backgroundcolor=\color{lbcolor},
	tabsize=4,
	rulecolor=,
	language=java,
        basicstyle=\scriptsize,
        upquote=true,
        aboveskip={1.5\baselineskip},
        columns=fixed,
        showstringspaces=false,
        extendedchars=true,
        breaklines=true,
        prebreak = \raisebox{0ex}[0ex][0ex]{\ensuremath{\hookleftarrow}},
        frame=single,
        showtabs=false,
        showspaces=false,
        showstringspaces=false,
        identifierstyle=\ttfamily,
        keywordstyle=\color[rgb]{0,0,1},
        commentstyle=\color[rgb]{0.133,0.545,0.133},
        stringstyle=\color[rgb]{0.627,0.126,0.941},
}
\usepackage{graphicx}
\fancyhf{}
\fancyhead[LE,LO]{\leftmark}
\fancyfoot[CE,CO]{- \thepage\ -}
%\linespread{1.3}
\fancypagestyle{plain}{
\fancyhead[LE,LO]{\leftmark}
\fancyfoot[CE,CO]{- \thepage\ -} 
}
\raggedbottom


%**************************************************************************
% Dane do strony tytułowej
%

% Autor
\autor{Piotr Zyszczak,Artur Śnioszek,Damian Łukasik}

% Rodzaj pracy - wpisać LICENCJACKA, INŻYNIERSKA lub MAGISTERSKA
\rodzajPracy{}

% Tytuł pracy magisterskiej/inżynierskiej
\tytul{Prognoza pogody}

% Rok
\rok{2016}

% Kierunek
\kierunek{Informatyka}

% Studia stacjonarne lub niestacjonarne (wpisać jakie)
\studia{stacjonarne}

% Poziom studiów wpisać I lub II
\poziomStudiow{II}


% Numer albumu
\numerAlbumu{113066,113055,112993}

%
%**************************************************************************


% Styl dla wtrąceń anglojęzycznych
\newcommand{\eng}[1]{(\emph{#1})}

\begin{document}

\stronaTytulowa

\tableofcontents
\chapter{Cel i zakres projektu}
Zaimplementować program zawierający technologie takie jak:
\begin{itemize}
\item System okienkowy
\item Grafika rastrowa oparta o GDI, Directx lub OpenGL
\item Wielowątkowość
\item Połączenie do bazy danych SQL
\item Połączenie sieciowe i obsługa sieci na poziomie gniazd z przejściem układu I/O na system wiadomości  windows(R) lub nowy watek z obsługą komunikacji sieciowej w technologii z obsługą gniazd bez przejścia z układu I/O na wiadomości systemu windows(R) (winsock.dll)  
\end{itemize}
Zdecydowano się więc na serwis pogodowy, który będzie pobierał dane z iternetu dzięki zapytaniom http.Połączenie z bazą bedzie tylko demonstracją technologii.


\chapter{Wykorzystane technologie}
Windows API, lub krócej: WinAPI – interfejs programistyczny systemu Microsoft Windows – jest to zbiór funkcji, stałych i zmiennych umożliwiających działanie programu w systemie Microsoft Windows.
Zbiór ten jest obszerny i zawiera funkcje do tworzenia okien programów, elementów interfejsu graficznego, obsługi zdarzeń oraz umożliwiające dostęp do innych aplikacji, funkcji sieciowych czy sprzętu w komputerze. Mianem WinAPI określamy standardowe funkcje przychodzące wraz z plikami bibliotek DLL (w 16-bitowych wersjach z rozszerzeniem .EXE) dostarczanymi z systemem, np. kernel32.dll, user32.dll, gdi32.dll, wsock32.dll, znajdującymi się w katalogu /WINDOWS/system32. Liczba plików bibliotek wzrasta w nowszych wersjach systemu Microsoft Windows. Może to powodować pewne problemy z uruchomieniem aplikacji napisanej dla starszej wersji systemu. Ze względu na dużą popularność systemu Microsoft Windows, obecnie większość środowisk programistycznych posiada zaimplementowane odpowiednie pliki nagłówkowe umożliwiające korzystanie z WinAPI.

GDI \eng{ang. Graphics Device Interface} - komponent Microsoft Windows pozwalający na odwzorowanie grafiki na urządzeniach zewnętrznych, np. drukarkach i monitorach.

ODBC (\eng{ang. Open DataBase Connectivity} - otwarte łącze baz danych) - interfejs pozwalający programom łączyć się z systemami zarządzającymi bazami danych. Jest to API niezależne od języka programowania, systemu operacyjnego i bazy danych. Standard ten został opracowany przez SQL Access Group we wrześniu 1992 roku. W skład ODBC wchodzą wywołania wbudowane w aplikacje oraz sterowniki ODBC.
Pierwsza implementacja ODBC pojawiła się w systemie Microsoft Windows, lecz obecnie możliwe jest korzystanie z niego na platformach Unix, OS/2 oraz Macintosh.
W systemach bazodanowych typu klient-serwer (np. Oracle lub PostgreSQL) sterowniki dają dostęp do silnika baz danych, natomiast w programach dla komputerów osobistych sterowniki sięgają bezpośrednio do danych. Mechanizm ODBC współpracuje zarówno z bazami obsługującymi język SQL, jak i nieobsługującymi go - w tym ostatnim przypadku język SQL jest przekładany na oryginalny język bazy danych.

Definicje wzięte z serwisu wikipedia.com.
\chapter{Implementacja}
W tym rozdziale opiszę po kolei funkcje i opowiem jak realizują założenia. Opis nalezy zacząć od głównej funkcji programu która wygląda naztępująco:
		
\begin{figure}[H]
\centering
\begin{lstlisting}[frame=single]		
int WINAPI WinMain(HINSTANCE hInstance, HINSTANCE hPrevInstance, LPSTR lpCmdLine, int nCmdShow) {
	InitializeCriticalSection( & g_Section );
	InitializeCriticalSection( & g_Section1 );
	WNDCLASSEX wc; /* A properties struct of our window */
	MSG msg; /* A temporary location for all messages */

	/* zero out the struct and set the stuff we want to modify */
	memset(&wc,0,sizeof(wc));
	wc.cbSize		 = sizeof(WNDCLASSEX);
	wc.lpfnWndProc	 = WndProc; /* This is where we will send messages to */
	wc.hInstance	 = hInstance;
	wc.hCursor		 = LoadCursor(NULL, IDC_ARROW);
	
	/* White, COLOR_WINDOW is just a #define for a system color, try Ctrl+Clicking it */
	wc.hbrBackground = (HBRUSH)(COLOR_WINDOW+1);
	wc.lpszClassName = "WindowClass";
	wc.hIcon		 = LoadIcon(NULL, IDI_APPLICATION); /* Load a standard icon */
	wc.hIconSm		 = LoadIcon(NULL, IDI_APPLICATION); /* use the name "A" to use the project icon */
	
/* zero out the struct and set the stuff we want to modify */
					memset(&wc1,0,sizeof(wc1));
					wc1.cbSize		 = sizeof(WNDCLASSEX);
					wc1.lpfnWndProc	 = WndProc1; /* This is where we will send messages to */
					wc1.hInstance	 = hInstance;
					wc1.hCursor		 = LoadCursor(NULL, IDC_ARROW);
	
					/* White, COLOR_WINDOW is just a #define for a system color, try Ctrl+Clicking it */
					wc1.hbrBackground = (HBRUSH)(COLOR_WINDOW+1);
					
\end{lstlisting}
\caption{Główna funkcja.}%
\label{rys:etykieta}
\end{figure}	

\begin{figure}[H]
\centering
\begin{lstlisting}[frame=single]	
					wc1.lpszClassName = "WindowClass1";
					wc1.hIcon		 = LoadIcon(NULL, IDI_APPLICATION); /* Load a standard icon */
					wc1.hIconSm		 = LoadIcon(NULL, IDI_APPLICATION); /* use the name "A" to use the project icon */
					
					/* zero out the struct and set the stuff we want to modify */
					memset(&wc2,0,sizeof(wc2));
					wc2.cbSize		 = sizeof(WNDCLASSEX);
					wc2.lpfnWndProc	 = WndProc2; /* This is where we will send messages to */
					wc2.hInstance	 = hInstance;
					wc2.hCursor		 = LoadCursor(NULL, IDC_ARROW);
					/* White, COLOR_WINDOW is just a #define for a system color, try Ctrl+Clicking it */
					wc2.hbrBackground = (HBRUSH)(COLOR_WINDOW+1);
					wc2.lpszClassName = "WindowClass2";
					wc2.hIcon		 = LoadIcon(NULL, IDI_APPLICATION); /* Load a standard icon */
					wc2.hIconSm		 = LoadIcon(NULL, IDI_APPLICATION); /* use the name "A" to use the project icon */


	
	if(!RegisterClassEx(&wc)) {
		MessageBox(NULL, "Window Registration Failed!","Error!",MB_ICONEXCLAMATION|MB_OK);
		return 0;
	}
	
	if(!RegisterClassEx(&wc1)) {
		MessageBox(NULL, "Window Registration Failed!","Error!",MB_ICONEXCLAMATION|MB_OK);
		return 0;
	}
	
	if(!RegisterClassEx(&wc2)) {
		MessageBox(NULL, "Window Registration Failed!","Error!",MB_ICONEXCLAMATION|MB_OK);
		return 0;
	}
	
	
	\end{lstlisting}
\caption{Główna funkcja.}%
\label{rys:etykieta}
\end{figure}	

\begin{figure}[H]
\centering
\begin{lstlisting}[frame=single]	
hwnd = CreateWindowEx(WS_EX_CLIENTEDGE,"WindowClass","Okno glowne",WS_VISIBLE|WS_OVERLAPPEDWINDOW,
		CW_USEDEFAULT, /* x */
		CW_USEDEFAULT, /* y */
		420, /* width */
		350, /* height */
		NULL,NULL,hInstance,NULL);

	if(hwnd == NULL) {
		MessageBox(NULL, "Window Creation Failed!","Error!",MB_ICONEXCLAMATION|MB_OK);
		return 0;
	}
	hwnd2 = CreateWindowEx(WS_EX_CLIENTEDGE,"WindowClass1","Prognoza",WS_VISIBLE|WS_OVERLAPPEDWINDOW,
		CW_USEDEFAULT, /* x */
		CW_USEDEFAULT, /* y */
		420, /* width */
		350, /* height */
		NULL,NULL,hInstance,NULL);

	if(hwnd2 == NULL) {
		MessageBox(NULL, "Window Creation Failed!","Error!",MB_ICONEXCLAMATION|MB_OK);
		return 0;
	}
	
	hwnd3 = CreateWindowEx(WS_EX_CLIENTEDGE,"WindowClass2","Autorzy",WS_VISIBLE|WS_OVERLAPPEDWINDOW,
		CW_USEDEFAULT, /* x */
		CW_USEDEFAULT, /* y */
		420, /* width */
		350, /* height */
		NULL,NULL,hInstance,NULL);

	if(hwnd3 == NULL) {
		MessageBox(NULL, "Window Creation Failed!","Error!",MB_ICONEXCLAMATION|MB_OK);
		return 0;
	}

	GenerateButtons(hwnd, hInstance);
	GenerateButtonsWeather(hwnd2, hInstance);
	GenerateButtonsAuthors(hwnd3, hInstance);
	
	ShowWindow(hwnd2,SW_HIDE);
	ShowWindow(hwnd3,SW_HIDE);
\end{lstlisting}
\caption{Główna funkcja.}%
\label{rys:etykieta}
\end{figure}	

\begin{figure}[H]
\centering
\begin{lstlisting}[frame=single]	
	/*
		This is the heart of our program where all input is processed and 
		sent to WndProc. Note that GetMessage blocks code flow until it receives something, so
		this loop will not produce unreasonably high CPU usage
	*/
	while(GetMessage(&msg, NULL, 0, 0) > 0) { /* If no error is received... */
		TranslateMessage(&msg); /* Translate key codes to chars if present */
		DispatchMessage(&msg); /* Send it to WndProc */
	}
	DeleteCriticalSection(& g_Section);
	DeleteCriticalSection(& g_Section1);
	return msg.wParam;
}
\end{lstlisting}
\caption{Główna funkcja.}%
\label{rys:etykieta}
\end{figure}		

W funkcji głównej znajdziemy przede wszystkim informacje o liczbie okien, kolorze tła, rozmiarze, nagłówkach. Są tam też odniesienia do funkcji gdzie znajdują się reakcje na zdarzenia.

\begin{figure}[H]
\centering
\begin{lstlisting}[frame=single]	
LRESULT CALLBACK WndProc(HWND hwnd, UINT Message, WPARAM wParam, LPARAM lParam) {
	switch(Message) {
		/* Upon destruction, tell the main thread to stop */
		case WM_CLOSE: {
			switch(MessageBox(NULL, "Chcesz zamknac?","Zamykanie?",MB_ICONQUESTION|MB_YESNO)){
						case IDYES:{
							PostQuitMessage(0);
							break;
						}
						case IDNO:{
							MessageBox(NULL, "Nie!","Ups",MB_ICONINFORMATION|MB_OK);
							break;
						}
			}
			break;
		}
		
		case WM_COMMAND: {
			switch(wParam) {
				case B_Option1: {
				
				ShowWindow(hwnd2,SW_SHOW);	
					
					break;
				}
				case B_Option2: {
					
				ShowWindow(hwnd3,SW_SHOW);	
				
					break;	
				}
				case B_Option3: {
					switch(MessageBox(NULL, "Chcesz zamknac?","Zamykanie?",MB_ICONQUESTION|MB_YESNO)){
						case IDYES:{
							PostQuitMessage(0);
							break;
						}
						case IDNO:{
							MessageBox(NULL, "Nie!","Odmowa",MB_ICONINFORMATION|MB_OK);
							break;
\end{lstlisting}
\caption{Główne menu.}%
\label{rys:etykieta}
\end{figure}	

\begin{figure}[H]
\centering
\begin{lstlisting}[frame=single]	
												}
					break;
					}
				}
				default: {
					break;
				}
			}
			break;
		}
		case WM_PAINT: {
			OnPaint(hwnd);
			break;
		}
		
		/* All other messages (a lot of them) are processed using default procedures */
		default:
			return DefWindowProc(hwnd, Message, wParam, lParam);
	}
	return 0;
}
\end{lstlisting}
\caption{Główne menu.}%
\label{rys:etykieta}
\end{figure}

Jak można zauważyć powyżej zdażenia są przechwytywane i rozpatrywane za pomocą funkcji switch. Funkcja ta rozpoznaje wcześniej zdefiniowaną nazwę przycisku i reaguje w sosób zdefiniowany przez programistę dla konkretnego przycisku. Można też dostrzec iż mechanizm okienkowy został zaimplementowany przy pomocy funkcji chowających i pokazujących okna gdyż nie było potrzeby ich tworzenia przy każdym naciśnięciu przycisku. Z tejże funkcji uruchamiana jest też funkcja rysująca przykłądową grafikę.

\begin{figure}[H]
\centering
\begin{lstlisting}[frame=single]	
LRESULT OnPaint(HWND hwnd){
	PAINTSTRUCT ps;
	HDC hdc;
	int l=0;
	//static int x,y;

	hdc = BeginPaint(hwnd, &ps);
	RECT rect;
	GetClientRect(hwnd, &rect);
	
	HBRUSH brush = CreateSolidBrush(RGB(0,0,255));
	//FillRect(hdc, &rect, brush);
	SelectObject(hdc, brush);
	
	
	//Rainy claud
	Ellipse(hdc ,200,150,250,200 );
	Ellipse(hdc ,230,150,280,200 );
	Ellipse(hdc ,260,150,310,200 );
	Ellipse(hdc ,290,150,340,200 );
	l+=10;//+l
	Ellipse(hdc ,200+l,150+l,250+l,200 +l);
	Ellipse(hdc ,230+l,150+l,280+l,200 +l);
	Ellipse(hdc ,260+l,150+l,310+l,200+l );
	Ellipse(hdc ,290+l,150+l,340+l,200+l );
	MoveToEx(hdc, rect.left + 210, rect.top + 220, NULL);
	LineTo(hdc, rect.left + 210, rect.top + 250);
	MoveToEx(hdc, rect.left + 230, rect.top + 220, NULL);
	LineTo(hdc, rect.left + 230, rect.top + 250);
	MoveToEx(hdc, rect.left + 250, rect.top + 220, NULL);
	LineTo(hdc, rect.left + 250, rect.top + 250);
	MoveToEx(hdc, rect.left + 270, rect.top + 220, NULL);
	LineTo(hdc, rect.left + 270, rect.top + 250);
	MoveToEx(hdc, rect.left + 290, rect.top + 220, NULL);
	LineTo(hdc, rect.left + 290, rect.top + 250);
	MoveToEx(hdc, rect.left + 310, rect.top + 220, NULL);
	LineTo(hdc, rect.left + 310, rect.top + 250);
	MoveToEx(hdc, rect.left + 330, rect.top + 220, NULL);
	LineTo(hdc, rect.left + 330, rect.top + 250);
	
	EndPaint(hwnd, &ps);
	DeleteObject(brush);
}
\end{lstlisting}
\caption{Funkcja graficzna.}%
\label{rys:etykieta}
\end{figure}

Funkcja ta przy pomocy komend graficznych rysuje chmurę za pomocą prostych kształtów (linii i kółek) co realizuje jeden z punktów.

\begin{figure}[H]
\centering
\begin{lstlisting}[frame=single]	
void GenerateButtons(HWND parent, HINSTANCE hInstance){
	CreateWindow(TEXT("STATIC"), TEXT("Witaj w programie Prognoza Pogody."),
                              WS_CHILD | WS_VISIBLE ,
                              10, 10, 350, 25,
                              parent, (HMENU)(502),
                              hInstance, NULL);
		
	CreateWindowEx(WS_EX_CLIENTEDGE,"Button","Wyszukaj Pogode",WS_VISIBLE|WS_CHILD|BS_PUSHBUTTON,
		50, /* x */
		50, /* y */
		130, /* width */
		30, /* height */
		parent,(HMENU)B_Option1,hInstance,NULL);
	
	CreateWindowEx(WS_EX_CLIENTEDGE,"Button","Autorzy",WS_VISIBLE|WS_CHILD|BS_PUSHBUTTON,
		50, /* x */
		90, /* y */
		70, /* width */
		30, /* height */
		parent,(HMENU)B_Option2,hInstance,NULL);
		
	CreateWindowEx(WS_EX_CLIENTEDGE,"Button","Koniec",WS_VISIBLE|WS_CHILD|BS_PUSHBUTTON,
		50, /* x */
		130, /* y */
		70, /* width */
		30, /* height */
		parent,(HMENU)B_Option3,hInstance,NULL);
}
\end{lstlisting}
\caption{Funkcja tworząca kontrolki do menu głównego.}%
\label{rys:etykieta}
\end{figure}
Przyciski powstały w osobnej funkcji by ułatwić znalezienie ich.
\begin{figure}[H]
\centering
\begin{lstlisting}[frame=single]	
LRESULT CALLBACK WndProc2(HWND hwnd, UINT Message, WPARAM wParam, LPARAM lParam) {
	switch(Message) {
		
		/* Upon destruction, tell the main thread to stop */
		case WM_CLOSE: {
			switch(MessageBox(NULL, "Chcesz zamknac?","Zamykanie?",MB_ICONQUESTION|MB_YESNO)){
						case IDYES:{
							ShowWindow(hwnd,SW_HIDE);
							break;
						}
						case IDNO:{
							MessageBox(NULL, "Nie!","Ups",MB_ICONINFORMATION|MB_OK);
							break;
						}
					}
			break;
		}
		
		case WM_COMMAND: {
			switch(wParam) {
				case Closing: {
					ShowWindow(hwnd,SW_HIDE);
					//MessageBox(NULL, "Nie!","Odmowa",MB_ICONINFORMATION|MB_OK);
					break;
				}
			}
			break;
		}
		
		/* All other messages (a lot of them) are processed using default procedures */
		default:
			return DefWindowProc(hwnd, Message, wParam, lParam);
	}
	return 0;
\end{lstlisting}
\caption{Spis autorów.}%
\label{rys:etykieta}
\end{figure}
Zdarzenia dla okna wypisującego autorów programu.
\begin{figure}[H]
\centering
\begin{lstlisting}[frame=single]	
void GenerateButtonsAuthors(HWND parent, HINSTANCE hInstance){
	
	//static HWND hwnd_ed_u;
	CreateWindow(TEXT("STATIC"), TEXT("Mamy nastepujacy sklad:"),
                              WS_CHILD | WS_VISIBLE ,
                              50, 10, 200, 25,
                              parent, (HMENU)(502),
                              hInstance, NULL);
							  	
	CreateWindow(TEXT("STATIC"), TEXT("inz. Piotr Zyszczak"),
                              WS_CHILD | WS_VISIBLE ,
                              50, 50, 200, 25,
                              parent, (HMENU)(502),
                              hInstance, NULL);
                              
    CreateWindow(TEXT("STATIC"), TEXT("inz Artur Snioszek"),
                              WS_CHILD | WS_VISIBLE ,
                              50, 90, 200, 25,
                              parent, (HMENU)(502),
                              hInstance, NULL);   
                              
	CreateWindow(TEXT("STATIC"), TEXT("inz Damian Lukasik"),
                              WS_CHILD | WS_VISIBLE ,
                              50, 130, 200, 25,
                              parent, (HMENU)(502),
                              hInstance, NULL);  
                              
	CreateWindowEx(WS_EX_CLIENTEDGE,"Button","Zamknij",WS_VISIBLE|WS_CHILD|BS_PUSHBUTTON,
		50, /* x */
		170, /* y */
		130, /* width */
		30, /* height */
		parent,(HMENU)Closing,hInstance,NULL);                      
}
\end{lstlisting}
\caption{Kontrolki do spisu autorów.}%
\label{rys:etykieta}
\end{figure}
Funkcja robi to samo co tworząca kontrolki dla menu.

\begin{figure}[H]
\centering
\begin{lstlisting}[frame=single]	
LRESULT CALLBACK WndProc1(HWND hwnd, UINT Message, WPARAM wParam, LPARAM lParam) {
	
	
	int Data_Of_Thread_1 = 1;
	int Data_Of_Thread_2 = 1;
	HANDLE Array_Of_Thread_Handles[3];
	switch(Message) {
		
		/* Upon destruction, tell the main thread to stop */
		case WM_CLOSE: {
			switch(MessageBox(NULL, "Chcesz zamknac?","Zamykanie?",MB_ICONQUESTION|MB_YESNO)){
						case IDYES:{
							ShowWindow(hwnd,SW_HIDE);
							break;
						}
						case IDNO:{
							MessageBox(NULL, "Nie!","Ups",MB_ICONINFORMATION|MB_OK);
							break;
						}
			}
			break;
		}
		
		case WM_COMMAND: {
			switch(wParam) {
				case Closing2: {
					ShowWindow(hwnd,SW_HIDE);
					break;
				}
				case Chconn: {
					//InitializeCriticalSection( & g_Section );
					HANDLE Handle_Of_Thread_1 = CreateThread( NULL, 0,FunkcjaConnectowa, &Data_Of_Thread_1, 0, NULL);
					//Array_Of_Thread_Handles[0] = Handle_Of_Thread_1;
					//WaitForSingleObject( Handle_Of_Thread_1, 500);
					DWORD rs = WaitForSingleObject( Handle_Of_Thread_1, 10000);
				
					
					if(rs == WAIT_OBJECT_0)
					{
						MessageBox(NULL,"Watek zakonczyl sie","Komunikat",MB_ICONINFORMATION|MB_OK);	 
					}
					
\end{lstlisting}
\caption{Zdarzenia okna prezentującego pogodę.}%
\label{rys:etykieta}
\end{figure}
\begin{figure}[H]
\centering
\begin{lstlisting}[frame=single]
					else if(rs == WAIT_TIMEOUT)
					{
						MessageBox(NULL,"Przekroczono czas","Komunikat",MB_ICONINFORMATION|MB_OK);	        		
					}else if(rs == WAIT_FAILED)
					{
						MessageBox(NULL,"Funkcja nie powiodla sie","Komunikat",MB_ICONINFORMATION|MB_OK);	        		
					}
					else if(rs == WAIT_ABANDONED)
					{
						MessageBox(NULL,"Blad","Komunikat",MB_ICONINFORMATION|MB_OK);	  
					}
					//MessageBox(NULL,buffer,"Watek pogodowy",MB_ICONINFORMATION|MB_OK);
				
					//DeleteCriticalSection(& g_Section);
					CloseHandle(Handle_Of_Thread_1);
						
					if (StatusWatek1==-1) {
						MessageBox(NULL,"Watek nie uruchomiony","Komunikat",MB_ICONINFORMATION|MB_OK);
					}
					else if (StatusWatek1==0) {
						MessageBox(NULL,"Zakonczono pobieranie","Komunikat",MB_ICONINFORMATION|MB_OK);
					}
					else if (StatusWatek1==1) {
						MessageBox(NULL,"Nadal pobieram dane","Komunikat",MB_ICONINFORMATION|MB_OK);
					}
					else if(StatusWatek1==2) {
						MessageBox(NULL,"Wysylam zapytanie","Komunikat",MB_ICONINFORMATION|MB_OK);
					}
					else if(StatusWatek1==3) {
						MessageBox(NULL,"Blad w funkcji connect","Komunikat",MB_ICONINFORMATION|MB_OK);
					}
					
\end{lstlisting}
\caption{Zdarzenia okna prezentującego pogodę.}%
\label{rys:etykieta}
\end{figure}
\begin{figure}[H]
\centering
\begin{lstlisting}[frame=single]
				else if(StatusWatek1==4) {
						MessageBox(NULL,"Blad inicjacji wsastartup","Komunikat",MB_ICONINFORMATION|MB_OK);
					}
					else if(StatusWatek1==7) {
						MessageBox(NULL,"Nie ma Internetu","Komunikat",MB_ICONINFORMATION|MB_OK);
					}
					MessageBox(NULL,buffer_w1,"Komunikat",MB_ICONINFORMATION|MB_OK);
					memset(buffer_w1, 0, sizeof buffer_w1);
					break;
				}
				case DBtest: {
					//InitializeCriticalSection( & g_Section1 );
					HANDLE Handle_Of_Thread_2 = CreateThread( NULL, 0,FunkcjaBazodanowa, &Data_Of_Thread_2, 0, NULL);
					if (WaitForSingleObject( Handle_Of_Thread_2,100000) !=WAIT_TIMEOUT){
					if(StatusWatek2==0){
						MessageBox(NULL, "Blad w watku!","Blad!",MB_ICONINFORMATION|MB_OK);
					}
					if(StatusWatek2==1){
						MessageBox(NULL, dest_buf_w2,"Wszystko ok!",MB_ICONINFORMATION|MB_OK);
					}
					if(StatusWatek2==2){
						MessageBox(NULL, message_w2,"Blad wdostepie do bazy!",MB_ICONINFORMATION|MB_OK);
					}
					}
        			else{
    					MessageBox(NULL,"Proces przekroczyl czas!","Wszystko ok",MB_ICONINFORMATION|MB_OK);
					}
					memset(dest_buf_w2, 0, sizeof dest_buf_w2);
					memset(message_w2, 0, sizeof message_w2);
					CloseHandle(Handle_Of_Thread_2);
					break;
				}
			}
			break;
		}
		
		/* All other messages (a lot of them) are processed using default procedures */
		default:
			return DefWindowProc(hwnd, Message, wParam, lParam);
	}
	return 0;
}
\end{lstlisting}
\caption{Zdarzenia okna prezentującego pogodę.}%
\label{rys:etykieta}
\end{figure}
W pewnym sensie nowością w tym oknie jest zastosowanie wątków do wywołania funkcji które będą ciałami tych wątków (funkcja createThread). Dodatkowo można wspomnieć o mechaniźmie zmiennych oznaczających różne fazy wątku np.Brak internetu spowoduje awaryjne wyjście z funkcji w wątku wywołanym z kalwisza zdefiniowanego jako Chconn. W obu pewne dane trzeba było przekazać do głównego wątku. By te nie kolidowały ze sobą użyto mechanizmu sesji krytycznej.

\begin{figure}[H]
\centering
\begin{lstlisting}[frame=single]
int FunkcjaConnectowa(){
	
	char buffer[100000];
	buffer[0] = 0;
	
	
	char *mess;
	
	WSADATA wsaData;
    if (WSAStartup(MAKEWORD(2,2), &wsaData) != 0) {
       // MessageBox(NULL, "dsa","WSA startup failed",MB_ICONINFORMATION|MB_OK);
        EnterCriticalSection( & g_Section );
	 	StatusWatek1=4;
	 	LeaveCriticalSection( & g_Section );
        system("pause");
        return 1;
    }
  
    SOCKET Socket=socket(AF_INET,SOCK_STREAM,IPPROTO_TCP);
    struct hostent *host;
    
    host = gethostbyname("api.wunderground.com");
    if(host != NULL){
    }
    else{
		EnterCriticalSection( & g_Section );
	 	StatusWatek1=7;
	 	LeaveCriticalSection( & g_Section );
	 	system("pause");
	 	return 1;}
	//WSACleanup();

    SOCKADDR_IN SockAddr;
    SockAddr.sin_port=htons(80);
    SockAddr.sin_family=AF_INET;
    SockAddr.sin_addr.s_addr = *((unsigned long*)host->h_addr);
 \end{lstlisting}
\caption{Funkcja pobierająca dane z internetu.}%
\label{rys:etykieta}
\end{figure}
\begin{figure}[H]
\centering
\begin{lstlisting}[frame=single]   
    if(connect(Socket,(SOCKADDR*)(&SockAddr),sizeof(SockAddr)) != 0){
        EnterCriticalSection( & g_Section );
	 	StatusWatek1=3;
	 	LeaveCriticalSection( & g_Section );
        system("pause");
        return 1;
    }
    
   	DWORD dlugosc = GetWindowTextLength( hText );
	LPSTR Bufor =( LPSTR ) GlobalAlloc( GPTR, dlugosc + 1 );
	GetWindowText( hText, Bufor, dlugosc + 1 );
	
	DWORD dlugosc2 = GetWindowTextLength( hText2 );
	LPSTR Bufor2 =( LPSTR ) GlobalAlloc( GPTR, dlugosc2 + 1 );
	GetWindowText( hText2, Bufor2, dlugosc2 + 1 );
	
	
	char* char1=(char*)Bufor;
	char* char2=(char*)Bufor2;
	char* char3= "GET /api/5df3f8dcf842e4e7/geolookup/conditions/forecast/q/";
	char* char4= "/";
	char* char5= ".json HTTP/1.1\r\nHost: api.wunderground.com\r\n\r\n";
	char dest_buf[100]; 
	wsprintf (dest_buf, "%s%s", char3, char1);
	wsprintf (dest_buf, "%s%s", dest_buf, char4);
	wsprintf (dest_buf, "%s%s", dest_buf, char2);
	wsprintf (dest_buf, "%s%s", dest_buf, char5);

 \end{lstlisting}
\caption{Funkcja pobierająca dane z internetu.}%
\label{rys:etykieta}
\end{figure}
\begin{figure}[H]
\centering
\begin{lstlisting}[frame=single] 	
	mess = dest_buf;
    if(send(Socket , mess , strlen(mess) , 0) < 0)
    {
        EnterCriticalSection( & g_Section );
	 	StatusWatek1=2;
	 	LeaveCriticalSection( & g_Section );
    }
	
    int nDataLength;
    
   
    while ((nDataLength = recv(Socket,buffer,2000,0)) > 0){        
        
       // MessageBox(NULL, buffer, "Connecting",MB_ICONINFORMATION|MB_OK);
        EnterCriticalSection( & g_Section );
        wsprintf (buffer_w1, "%s%s", buffer_w1, buffer);
	 	StatusWatek1=1;
	 	LeaveCriticalSection( & g_Section );
    }
    
    //EnterCriticalSection( & g_Section );
   // recv(Socket,buffer,100000,0);
    //LeaveCriticalSection( & g_Section );
    
    EnterCriticalSection( & g_Section );
  
   // wsprintf(buffer_w1, "%s%s", buffer_w1, buffer);
    StatusWatek1=0;
    LeaveCriticalSection( & g_Section );
    //wsprintf (dest_buf, "%s%s", dest_buf, char5);
    
    closesocket(Socket);
        WSACleanup();
}
\end{lstlisting}
\caption{Funkcja pobierająca dane z internetu.}%
\label{rys:etykieta}
\end{figure}
Funkcja pobiera dane z serwisu za pomocą poleceń z biblioteki winsock i libws2 32.a. Dane są pobierane po wusłaniu zapytania do serwisu wundergroune w postaci http z danymi miesta dla którego chcemy dostać pogodę i klucza. By takowy klucz otrzymać trzeba się zarejestrować w serwisie. Sesją krytyczną zostały otoczone miejsca z których może kożystać tylko jeden wątek.
\begin{figure}[H]
\centering
\begin{lstlisting}[frame=single]
int FunkcjaBazodanowa(){
	char dest_buf[500];
	
	EnterCriticalSection( & g_Section1 );
	//dest_buf[0] = 0;
	StatusWatek2=0;
	LeaveCriticalSection( & g_Section1 );
	
    SQLHANDLE sqlenvhandle;    
    SQLHANDLE sqlconnectionhandle;
    SQLHANDLE sqlstatementhandle;
    SQLRETURN retcode;

    if(SQL_SUCCESS!=SQLAllocHandle(SQL_HANDLE_ENV, SQL_NULL_HANDLE, &sqlenvhandle))
        goto FINISHED;

    if(SQL_SUCCESS!=SQLSetEnvAttr(sqlenvhandle,SQL_ATTR_ODBC_VERSION, (SQLPOINTER)SQL_OV_ODBC3, 0)) 
        goto FINISHED;
    
    if(SQL_SUCCESS!=SQLAllocHandle(SQL_HANDLE_DBC, sqlenvhandle, &sqlconnectionhandle))
        goto FINISHED;

    SQLCHAR retconstring[1024];
    switch(SQLDriverConnect (sqlconnectionhandle, 
                NULL, 
                (SQLCHAR*)"DSN=mysqlster;", 
                SQL_NTS, 
                retconstring, 
                1024, 
                NULL,
                SQL_DRIVER_COMPLETE)){
        case SQL_SUCCESS_WITH_INFO:
            show_error(SQL_HANDLE_DBC, sqlconnectionhandle);
            break;
        case SQL_INVALID_HANDLE:
        case SQL_ERROR:
            show_error(SQL_HANDLE_DBC, sqlconnectionhandle);
            goto FINISHED;
        default:
            break;
    }
  \end{lstlisting}
\caption{Funkcja łącząca aplikację z bazą.}%
\label{rys:etykieta}
\end{figure}
    
  \begin{figure}[H]
\centering
\begin{lstlisting}[frame=single]      
    if(SQL_SUCCESS!=SQLAllocHandle(SQL_HANDLE_STMT, sqlconnectionhandle, &sqlstatementhandle))
        goto FINISHED;

    if(SQL_SUCCESS!=SQLExecDirect(sqlstatementhandle, (SQLCHAR*)"select * from testtable", SQL_NTS)){
        show_error(SQL_HANDLE_STMT, sqlstatementhandle);
        goto FINISHED;
    }
    else{
        char name[64];
        char address[64];
        char id[64];
        while(SQLFetch(sqlstatementhandle)==SQL_SUCCESS){
            SQLGetData(sqlstatementhandle, 1, SQL_C_CHAR, id, 64, NULL);
            SQLGetData(sqlstatementhandle, 2, SQL_C_CHAR, name, 64, NULL);
            SQLGetData(sqlstatementhandle, 3, SQL_C_CHAR, address, 64, NULL);
            
            //EnterCriticalSection( & g_Section1 );
            wsprintf (dest_buf, "%s%s", dest_buf, id);
            wsprintf (dest_buf, "%s%s", dest_buf, name);
            wsprintf (dest_buf, "%s%s", dest_buf, address);
            
            EnterCriticalSection( & g_Section1 );
	StatusWatek2=1;
	wsprintf(dest_buf_w2, "%s%s", dest_buf_w2, dest_buf);
	LeaveCriticalSection( & g_Section1 );	
        }
    }
	FINISHED:
	SQLFreeHandle(SQL_HANDLE_STMT, sqlstatementhandle );
	SQLDisconnect(sqlconnectionhandle);
    SQLFreeHandle(SQL_HANDLE_DBC, sqlconnectionhandle);
    SQLFreeHandle(SQL_HANDLE_ENV, sqlenvhandle);
    
}
\end{lstlisting}
\caption{Funkcja łącząca aplikację z bazą.}%
\label{rys:etykieta}
\end{figure}

Funkcja łączy aplikację z bazą za pomocą sterownika odbc. By takowy zastosować potrzeba było bibliotek libodbc32.a, libodbccp32.a. Trzeba dodatkowo w systemie dodać ustawienie w panelu sterowania dla odbc.
\begin{figure}[H]
\centering
\begin{lstlisting}[frame=single]  
void show_error(unsigned int handletype, const SQLHANDLE handle){
    SQLCHAR sqlstate[1024];
    SQLCHAR message[1024];
    if(SQL_SUCCESS == SQLGetDiagRec(handletype, handle, 1, sqlstate, NULL, message, 1024, NULL)){
    	EnterCriticalSection( & g_Section1 );
    	StatusWatek2=2;
    	wsprintf (message_w2, "%s%s", message_w2, message);
        //cout<<"Message: "<<message<<"nSQLSTATE: "<<sqlstate<<endl;
        LeaveCriticalSection( & g_Section1 );
	}	
}  
\end{lstlisting}
\caption{Dodatkowa funkcja z komunikatem błędu.}%
\label{rys:etykieta}
\end{figure}
Ta funkcja wysyła do wątku głównego szczegółowy komunikat błędu. Może to być w postaci kodu 08001.
\begin{figure}[H]
\centering
\begin{lstlisting}[frame=single]  
#ifndef WIN32_LEAN_AND_MEAN
#define WIN32_LEAN_AND_MEAN
#endif
#include <string.h>
#include <windows.h>
#include <winsock2.h>
#include <ws2tcpip.h>
#include <iphlpapi.h>
#include <sqltypes.h>
#include <sql.h> 
#include <sqlext.h>
#pragma comment(lib, "libws2_32.a")
#pragma comment(lib, "libodbc32.a")
#pragma comment(lib, "libodbccp32.a")
#define BUFFERSIZE 1024
#define Label 	  99
#define B_Option1 100
#define B_Option2 101
#define B_Option3 102
#define TI_Edit   103 //Kraj
#define TI_Edit1   110 //Miasto
#define Closing   104
#define Closing2  104
#define Chconn    105
#define DBtest    106
	WNDCLASSEX wc1;
	WNDCLASSEX wc2;
	HWND hwnd;
	HWND hwnd2;
	HWND hwnd3;
	HANDLE Handle_Of_Thread_1 = 0;
	HWND hText,hText2;
	CRITICAL_SECTION g_Section;
	int StatusWatek1=-1;   
	CRITICAL_SECTION g_Section1;
	int StatusWatek2=-1;		
	char buffer1[1024];
	char buffer_w1[100000]; // dane z watku1 
	char dest_buf_w2[500]; // dane z watku2
	SQLCHAR message_w2[500]; // komunikat bledu w2
\end{lstlisting}
\caption{Biblioteki,zdefiniowane kontrolki i zmienne globalne.}%
\label{rys:etykieta}
\end{figure}

Biblioteki,zdefiniowane kontrolki i zmienne globalne zastosowane w programie. Dodatkowo trzeba było dociągnąć biblioteki zewnętrzne podłączone komendą pragma comment().

\chapter{Opis użycia}
Po uruchomieniu programu ukazuje nam się menu główne. Pierwsza opcja zabierze nas do ekranu gdzie możemy zdobyć informacje o pogodzie w dowolnym mieście na ziemi. Druga wyświetli listę autorów. Trzecia zakończy program.

\begin{figure}[H]
\centering
\includegraphics[scale=0.70]{menu.jpg}
\caption{Menu główne.}%
\label{rys:etykieta}
\end{figure} 
Ekran z autorami zawiera elementy typu label z danymi autorów programu (tytuł, imię i nazwisko). Dodatkowo jest przycik zamykający ten ekran.
\begin{figure}[H]
\centering
\includegraphics[scale=0.70]{autorzy.jpg}
\caption{Menu główne.}%
\label{rys:etykieta}
\end{figure} 
Nestępne okno zawiera dwie kontrolki z edycją tekstu gdzie można zgodnie z opisem wprowadzić dane potrzebne do zapytania http. Przycisk Sprawdź pogodę wyświetki nam raport z danymi pogodowymi i informacjami o połączeniu. Przycisk testuj bazę łączy z bazą MySQL i  zwraca pobrane z niej rekordy.
\begin{figure}[H]
\centering
\includegraphics[scale=0.70]{serwis.jpg}
\caption{Menu główne.}%
\label{rys:etykieta}
\end{figure} 
\chapter{Podsumowanie}
Zrealizowano wszystkie założenia projektu:
\begin{itemize}
\item System okienkowyzaimplementowano zgodnie z zaleceniami na zajęciach.
\item Grafika rastrowa zostałą stworzona w oparciu o GDI.
\item Wielowątkowość zaimplementowano w postaci dwóch dodatkowych wątków na połączenie z bazą i pobieranie danych ze strony.
\item Połączono z bazą danych MySQL przy pomocy sterownika ODBC.
\item Zastosowano wątek z obsługą komunikacji sieciowej w technologii z obsługą gniazd bez przejścia z układu I/O na wiadomości systemu windows(R) (winsock.dll)  
\end{itemize}
\end{document}
